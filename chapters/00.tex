\documentclass[output=paper]{langsci/langscibook} 
\title{Interpreting and technology: The upcoming technological turn} 
\author{Claudio Fantinuoli\affiliation{Mainz University}}
% \chapterDOI{} %will be filled in at production

% \epigram{Change epigram in chapters/01.tex or remove it there }

\abstract{\noabstract}

\shorttitlerunninghead{Interpreting and technology: The upcoming technological turn}

\begin{document}
\maketitle

%%please move the includegraphics inside the {figure} environment
%%\includegraphics[width=\textwidth]{Ch1-img1}

 
%%please move the includegraphics inside the {figure} environment
%%\includegraphics[width=\textwidth]{Ch1-img2}

 
%%please move the includegraphics inside the {figure} environment
%%\includegraphics[width=\textwidth]{Ch1-img3}
\section{Introduction} 
 
The topic of technology is not new in the context of interpreting. However, recent advances in interpreting-related technologies are attracting increasing interest from both scholars and practitioners. This volume aims at exploring key issues, approaches and challenges in the interplay of interpreting and technology, a domain of investigation that is still underrepresented in the field of Interpreting Studies. The contributions to this volume focus on topics in the area of computer-assisted and remote interpreting, both in the conference as well as in the court setting, and report on experimental studies.
 
To the best of my knowledge, this is the first book entirely dedicated to this subject. Its publication should not be considered a point of arrival in research work on interpretation and technology, but rather as an occasion to give new momentum to the analysis of a topic that is both current and complex. In this field further in-depth research is necessary in order to better understand the past and future impact of technology on interpretation, on the one hand, and to prepare future generations of interpreters to adapt to a constantly changing market, on the other.
 
\section{Setting technology into the interpreting perspective} 
When compared to written translation or other language professions, the advances in information and communication technology have had a modest impact on interpreting so far. In its long history, however, interpreting has not been immune to technological innovations. On the contrary, it has gone through at least two major technological breakthroughs with disruptive effects on the profession in both cases. 
 
The first breakthrough was the introduction of wired systems for speech transmission that led to the rise of simultaneous interpreting (\textsc{si}). First attempts in this direction were reported in the early 1920s, with a patent filed by IBM and its adoption at the Sixth Congress of the Comintern in the former Soviet Union and at the International Labour Conference. This technology acquired broader visibility during the post war Nuremberg trials and was adopted since then in all international organizations. Although the cognitive process of translating while listening to the source speech was not new (chuchotage has been probably around forever), the invention of simultaneous interpretation equipment radically changed the way interpretation was delivered on a daily basis. This technological breakthrough also had an impact on social status and self-perception of interpreters. At the beginning, interpreters feared a loss of quality in their performance and perceived the relegation into interpreting booths and the need to abandon the stage they used to share with diplomats as a worsening of the prestige associated to the profession and, consequently, of their social status. In reality, the broad adoption of \textsc{si} together with the increasing demand for interpreting services due to geopolitical changes in the second half of the 20th century led to a professionalization of the whole sector and, in turn, to a general improvement of the occupational status of interpreters.\footnote{For a brief history of interpreting, see for example \citeauthor{takeda_new_2016} (\citeyear{takeda_new_2016}) while for an overview of the social-status of simultaneous interpreters, see \citeauthor{Gentile2013} (\citeyear{Gentile2013}).} 
 
The second technological breakthrough that has affected interpreting practice is the Internet. The emerging of the Web in the 1990s radically changed interpreters’ relation to knowledge and its acquisition. Since preparation is one of the fundamental aspects of interpreting \citep{Gile2009}, as it is crucial to fill the linguistic and knowledge gap between event participants and interpreters, the impact of this technology on the profession has been extraordinary. The Web is the most comprehensive and accessible repository of textual material available in many languages and on many topics. Interpreters use it in a lot of different ways, for example to conduct exploratory research before they receive actual conference material \citep{chang_conference_2018}, to create specialized corpora for linguistic analyses (\citealt{Fantinuoli2017a,Fantinuoli2018c}; \citealt{xu_corpus-based_2018}) or simply to find translations for specialized terms.\largerpage[-1]

Search engines, in particular, have become the privileged door to knowledge \citep{finn_what_2017}. They are used to discern right from wrong, good from bad, or, in the limited scope of interpreting, to fill knowledge gaps, confirm translation hypotheses, find definitions, and so forth. Thanks to the undeniable advantages of having this wealth of information available with a simple click of the mouse, the Web has become by right the most familiar working environment for translators and interpreters \citep{zanettin_corpora_2002}. The Web (and digital devices) has changed not only our habits, but has influenced also our cognitive behaviour, for example through the modification of our reading patterns. Different form printed documents, which are commonly read line by line, digital documents are mostly scanned through in search for key terms or to get a general overview \citep{pernice_f-shaped_2017}. Since there is evidence that this change influences aspects of learning such as recall, comprehension and retention of knowledge \citep{ross_print_2017}, it is reasonable to assume that the digitization of information has had consequences on interpreting and its underlying subprocesses, especially in the pre-event phase of preparation. The magnitude of this change, however, is still not completely understood as no empirical  investigation has been  carried out so far to assess this in interpreting.%%FK: Separated Xu citekey from the Fantinuoli keys, because joining them produced an unintelligible error.
 
Currently, interpreting might be on the verge of a third breakthrough which I will call, for lack of a better term, the technological turn in interpreting. Bigger by one order of magnitude if compared to the first two breakthroughs, its pervasiveness and the changes that it may bring about could reach a level that has the potential to radically change the profession.\footnote{Referring to Hegel, \citeauthor{galimberti_i_2009} affirms that ``When a phenomenon grows \textit{quantitatively}, there is not only an increase in quantity, but there is also a radical change in \textit{quality}. Hegel provides a very simple example: if I pull out one hair, I am one who has hair, if I pull out two hairs I am one who has hair, if I pull out all of my hair I am bald. There is, therefore, a qualitative change for the simple quantitative increase of a gesture'' (\citeyear[215]{galimberti_i_2009}) (translation by the author).} Not only could this lead to a transformation of the interpreting ecosystem in all its complexity, but it is reasonable to assume that it may have a significant impact on many socio- economic aspects related to the profession, from the way it is perceived by the general public to the status and working conditions of interpreters. In order to explore the reasons for and the potential consequences of this technological turn, it is first necessary to briefly introduce the interpreting-related technologies that lie at the core of this discussion. 
 
\section{Interpreting-related technologies} 
There are three main areas that will play a central role in this technological turn: computer-assisted (\textsc{cai}), remote (\textsc{ri}), and machine interpreting (\textsc{mi}). 
 
Computer-assisted interpreting can be defined as a form of oral translation in which a human interpreter makes use of computer software designed to support and facilitate some aspects of the interpreting task with the goal to increase quality and -- to a minor extend -- productivity \citep{Fantinuoli2018}. Among others, \textsc{cai} tools are designed to assist interpreters in the creation of glossaries by means of integrating a wide range of terminology resources, in looking up terms or entities in an ergonomic way, and in extracting useful information from preparatory documents, to name but a few. They can make use of advanced Natural Language Process features, such as automatic terminology extraction, key topics identification, summarization, automatic speech recognition, and so forth.\footnote{For some examples of advanced use of Natural Language Processing applications in \textsc{cai} tools, see 
% \citeauthor{fantinuoli_speech_2017} (\citeyear{fantinuoli_speech_2017}
\citet{Fantinuoli2017b}
and \citet{stewart_automatic_2018}.} 
 
The most evident reason behind the creation of \textsc{cai} tools is the ambition to improve the interpreters’ work experience, by relieving them of the burden of some of the most time-consuming tasks (such as the creation and organization of terminology) and by supporting them in carrying out numerous activities, from the retrieval of preparatory documents to their analysis in a way appropriate to their profession. By improving the working experience of interpreters, both during preparation and during the very act of interpreting, \textsc{cai} tools ultimately aim at increasing the quality of the interpreting performance. Being an integral part of the interpreting process (suffice it to think of the most extreme case of accessing terminological information during simultaneous interpretation), they are directly linked to and may have an influence on the cognitive processes underlying the central tasks of interpreting.   
 
Remote interpreting is a broad concept which is commonly used to refer to forms of interpreter-mediated communication delivered by means of information and communication technology. It is not a monolithic notion, but it can rather be used to designate different settings and modalities, for example when all event participants are gathered at one place while the interpreters are located at a different venue, or when the interpreter and one of the interlocutors are both present at the same place. As far as technology is concerned, \textsc{ri} can be carried out by means of different solutions, from simple telephone to advanced videoconference equipment. 
 
Up until now \textsc{ri} has been used mainly to provide remote consecutive interpreting services, for example in the healthcare or judicial sector, while in other contexts, such as conference interpreting, \textsc{ri} has been scarcely deployed.\footnote{One notable exception is the use of \textsc{ri} in television interpretation.} The limited adoption of \textsc{ri} has to do both with limitations of the technologies available and with the complex cognitive and communicative processes underlying interpreting. Tests conducted on remote simultaneous interpreting (\textsc{rsi}), for instance, have highlighted, among others, issues in the quality of the audio/video signals, the partial loss of contextual information due to remoteness, and psychological factors, such as fatigue, higher levels of stress and loss of motivation and concentration. In the area of dialogue interpreting, issues like turn taking, alienation and stress have been found to be particularly significant.\footnote{For a bibliographical overview, see \citet{Andres2009}.}
 
Technological progress is, however, removing technical barriers to remote interpreting which is becoming a viable solution for many stakeholders in need to cut costs and increase service availability. The increasing demand for liaison and consecutive interpreting services, for example for refugees, has already led to the adoption of this technology by many public institutions\footnote{For an example of adoption in 2016 by the German Federal Office for Migration and Refugees, see http://www.bamf.de/DE/DasBAMF/BAMFdigital/Video-Dolmetscher/video-dolmetscher-node.html}. This may apply soon also to the context of simultaneous interpreting. Since empirical tests have shown that it is possible to perform, under certain circumstances, remote simultaneous interpretation without breaching professional associations’ codes, ISO standards or other related norms applicable to interpretation \citep[202]{Causo2011b}, the number of enterprises offering platforms for \textsc{ri} both in the form of interpreting hubs, i.e. professional environments with booths, high-quality consoles, technicians, etc., and in the form of solutions for home offices has dramatically increased. The scale of its adoption, however, is still unknown.   
 
Machine interpreting (\textsc{mi}), also known as automatic speech translation, automatic interpreting or speech-to-speech translation, is the technology that allows the translation of spoken texts from one language to another by means of a computer program. \textsc{mi} is a technology that aims at replacing human interpreters and is in this respect very different to the other two interpreting-related technologies, since they are designed to assist human interpreters in their work (\textsc{cai}) or to change the way they deliver their service (\textsc{ri}). It combines at least three technologies to perform the task: automatic speech recognition (\textsc{asr}), to transcribe the oral speech into written text, machine translation (\textsc{mt}), and speech-to-text synthesis (\textsc{stt}), to generate an audible version in the target language. 
 
Although \textsc{mi} is still very far from achieving  the ambitious promise of a comparable quality output as human interpreters, considerable improvements have been made over the last few years. This is due to the latest developments in several machine learning technologies: \textsc{asr} based on neural networks, for example, is quicker and more precise than ever while deep neural machine translation has reached unprecedented quality in terms of precision and fluency of the target language output. First prototypes of \textsc{mi} have been presented after long years of research in the field of natural language processing, such as the real-time automatic speech translation system for university lectures implemented at the Karlsruhe Institute of Technology \citep{muller_lecture_2016}, or have been brought on the market by technology giants, such as Google (Pixel Buds) or Microsoft (Skype Translator).  
 
The success of these systems has been quite modest so far as they fail to achieve the goal of quality and usability even for the most basic real scenarios in which interpreting is needed. The creation of machine interpreting systems is so challenging for several reasons, both at a technical and at a communicative level. On the technical side, quality of automatic translation and issues in the latency and flexibility of speech recognition as well as noise tolerance and speaker independence, to name but a few, exponentially increase the sources of errors and inaccuracies. On the communicative side, \textsc{MI} systems suffer from not being able to work – as yet - with cotext and context or to translate all the information that is not explicitly coded verbally, such as the speaker's attitude, world references, etc. However, the advances in machine learning are producing encouraging results not only in machine translation (resolving issues of lexical, syntactic, semantic and anaphoric ambiguity, to name but a few), but also in many related fields, such as sentiment analysis, attitude identification, and so forth. In the near future, the integration of these applications into \textsc{mi} may increase its quality, making it more ``intelligent'' and increasing its quality to a point where its use, at least in some contexts, could start to be conceivable.
 
\section{The upcoming technological turn} 
There is some evidence that the profession is heading towards a technological turn. First of all, the interpreting-related solutions brought about by new advances in information and communication technologies as well as in natural language processing are growing in number, and the speed of change is significantly faster than it was in the past. In the three areas indicated above, companies are investing time and effort in order to launch an ever increasing number of software and devices on the market, thus reacting to users’ demands but also creating new ones.
 
More important, however, is the fact that interpreting is caught up in fundamental and pervasive changes of the labor market due to technological developments, in particular to digitization and automation, which are creating new patterns of work organization \citep{ursula_huws_logged_2016,neufeind_work_2018}. Interpreting is not immune to these developments. Notwithstanding the relatively small economic impact of the interpreting sector,\footnote{Suffice it to compare interpreting with the written translation industry to see the importance of economic aspects in technology adoption. The cost-cutting potential of computer-assisted translation (\textsc{cat}) tools in the 1990s and, more recently, of machine translation in the translator’s workspace have forced the large-scale adoption of such tools, irrespective of the personal attitude of translators towards these innovations.} the pressure to embrace new technologies may soon increase. Not only the market, but also society, which, as \citeauthor{besnier_homme_2012} (\citeyear{besnier_homme_2012}) points out is literally obsessed with technology, may have enough persuasive power to impose a paradigm change on the profession, no matter the personal attitudes towards it and the concerns about potential consequences on quality, working conditions and so forth. If the technological adoption is quite unproblematic in the area of \textsc{cai} tools, as their use will influence only (micro)processes of the interpreting activity, but will not have any relevant socio-economic impact, for example on the labor market, the situation may become more complex as far as \textsc{ri} and \textsc{mi} are concerned. 

The real impact of these two technologies on the medium and long run is difficult to predict. In the case of \textsc{ri}, for example, there is no doubt that it will offer increased opportunities for work in new market segments, leading to a productivity effect, i.e. an increase in the demand for labor that arises due to technological progress. However, chances are that it may also lead to a deterioration of working conditions. The large-scale adoption of new interpreting-related technologies, such as \textsc{ri}, could drive a process of commoditization of interpretation, intensifying the effects of modern paradigms of labor organization, such as outsourcing (which is already typical in the language sector and many other professions of the tertiary sector). For example, it is plausible to think that \textsc{ri}, at least in some market segments, may bring about a partial depersonification of the service provider. When services become more impersonal and uniform from the buyers’ point of view, they tend to buy the cheapest, initiating a downward spiral of economic decline and, ultimately, de-professionalisation of the industry. 
 
\largerpage 
In this scenario, machine interpreting may further contribute to accelerate this process. Although \textsc{mi} is still in its infancy and the limits of current implementations are clear, there is no doubt that the fast evolution of this technology will have both a long-term impact in some areas of the profession (if/when the technology reaches a mature status it may put at risks interpreters jobs), and, most interestingly, a short-term impact in the public perception of the activity performed by professional interpreters (and consequently in the perception of different stakeholders). This, in turn, may under certain circumstances undermine the status of the profession well before the time  \textsc{mi} will actually represent a potential threat to human interpreters. 

It is probably for this or similar fears that interpreting technologies have been traditionally welcomed with a general attitude of aversion and skepticism by professionals. This hostility generally takes the form of arguments in defence of quality, in the case of \textsc{ri}, or in defence of the exclusive intellectual dimension of the interpreting activity, in the case of \textsc{cai}. Its real motivation is, however, the natural feeling of insecurity and fear of technologically induced changes and, consequently, the need to pursue a legitimate and strategic goal, the defence of the interests of the category \citep{pym_what_2011}. 
 
Paradoxically, a balanced and responsible adoption of technologies could be fruitful to reverse such negative trends. Looking at the broader picture, the most promising approach could be to try to use the technological advances for the benefit of the interpreters, reaping the benefits offered by technology and preventing the risk to be dominated by it. However, this requires at least two things. On the one hand, there is urgent need for a research effort directed to anticipating future trends, enabling the sector to prepare for the disruptive changes caused by digital technologies. This inquiry should not be conducted merely from the interpreter’s perspective (self-perception, etc.), albeit it remaining a crucial side in the debate, but it should also consider the interests of all stakeholders and encompass considerations of different nature, such as socio-economic parameters, etc. On the other, it requires the development of an open-minded attitude towards technology and the ability to rethink the profession as we know it today, on the basis of empirical evidence, new ideas and the awareness about the direction that markets, society and technological developments are heading to.

Even if still marginal in Interpreting Studies, it should be pointed out that the interest for technological matters, especially but not exclusively for \textsc{ri}, as well as the presence of technology in interpreter training are gaining momentum, indicating some degree of awareness is spreading in regards to the importance of technological development to interpreting. This is encouraging. The present book can be considered a small contribution in this direction as it offers some evidence, practical suggestions and new ideas that may help the interpreting community to positively address the upcoming challenges. All its chapters present empirical research in two areas of technological innovation which may have a greater impact on the daily working conditions of interpreters in the immediate future, namely computer-assisted and remote interpreting. 
 
\section{Overview of the individual contributions}\largerpage[-1]
 
The book opens with two seminal chapters in the research area around \textsc{cai} tools and should stimulate scientific and practical discussion on the role of technology use during interpreting. Desmet, Vandierendonck and Defrancq present a pilot study on the potential impact of \textsc{cai} tools that support the interpretation of numbers. The authors set up a mock-up system to simulate technology that automatically recognizes numbers in the source speech and presents them on a screen in the booth. The study experimentally shows that \textsc{cai} tools may have the potential to reduce the cognitive load during simultaneous interpreting and improve quality. Considering the quality reached by automatic speech recognition, this study may contribute to a faster adoption of this technology in the interpreting setting. 

The issue of finding the right framework to study the impact of \textsc{cai} tools on the interpreter delivery is pivotal in Prandi’s chapter. In her exploratory study, she evaluates the appropriateness of the stimuli adopted for data collection and describes the theoretical framework she chose to conduct the experiment. The final goal of this research project, still underway, is to verify whether the use of \textsc{cai} tools in the booth causes saturation or, on the contrary, helps prevent it by reducing the cognitive load during terminology search and delivery. The preliminary results derived from the analysis of the test subjects’ interpretations seems to indicate that the use of a \textsc{cai} tool, under specific circumstances, may increase output quality. 
 
Deysel and Lesch focus on \textsc{cait} and explore the use of such tools to develop self-assessment skills in the performance of professional interpreters working in the National Parliament of the Republic of South Africa. The research design for this article comprises an evaluation study approach, based on an experimental design that considers the exposure to \textsc{cait} for purposes of self-assessment. In order to collect data to address the research questions, a questionnaire, an experiment and interviews were used. The experimental group was exposed to the software \textit{Black Box} in order to measure its impact on the development of their self-assessment skills. The results show that the experimental group of practicing interpreters who were exposed to the software indicated a better understanding of the criteria which are important in the assessment of interpreting performance as well as a greater awareness of the strengths and weaknesses of their performance. 
 
Devaux’s chapter explores practicing court interpreters’ perceptions of their role in England and Wales when they interpret through videoconferencing systems. The author empirically approaches the subject conducting semi-structured interviews with eighteen participants. The data gathered was analyzed through the innovative theoretical framework of role-space. The results show that the use of technology, unlike in face-to-face court hearings, makes some interpreters perceive their role differently and forces them to create split role models. The use of videoconferencing equipment affects various aspects of their presentation of self, participant alignment, and interaction management. The chapter ends with some recommendations for training court interpreters derived from the experimental results. 
 
Finally, Ziegler and Gigliobianco address the use of remote interpreting in the simultaneous mode. After analyzing the terminological challenges and presenting the basic literature on the topic, they give a detailed overview of the state-of-the art of \textsc{ri}, the technical requirements required for remote interpreting and the relevant international norms. They then introduce a pilot experiment aiming at testing the feasibility of using augmented reality in order to overcome some of the perceived limitations of \textsc{ri}, i.e. exclusion and lack of visuality. The idea of interpreters working and being in control of what the camera(s) show them is certainly attractive and it may trigger research into new interpreting technologies applied to remoteness. 
 
\section{Conclusions} 
 
There seem to be signs of a new technological breakthrough approaching interpreting, yet not enough research and discussion is devoted to the actual consequences for the profession, both in the short and in the long term. There is an urgent need to understand how technology is disrupting the way interpreters work and to explore the broad terrain of private actions, public policies, and professional dialogue needed to ensure that technological advancements can be shaped to the benefit of interpreters. 

It is the hope of the editor that, through this publication, interpreting scholars and professionals will embrace further research and discussions in this exciting area of interpreting studies, exploring new topics at the intersection of technology and interpreting and, in doing so, contributing to preparing the profession to successfully face the upcoming technological turn in interpreting.  

\sloppy
\printbibliography[heading=subbibliography,notkeyword=this]

\end{document}
