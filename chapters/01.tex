\documentclass[output=paper]{langsci/langscibook} 
\ChapterDOI{10.5281/zenodo.1493291}
\title{Simultaneous interpretation of numbers and the impact of technological support} 
\author{Bart Desmet\affiliation{University of Ghent}\and Mieke Vandierendonck\affiliation{University of Ghent}\lastand Bart Defrancq\affiliation{University of Ghent}}

% \epigram{Change epigram in chapters/01.tex or remove it there }

\abstract{In simultaneous interpretation, numbers are a common source of errors. They are often characterized by low predictability from the context and high information density, and the interpreter is therefore required to change strategies with respect to listening, memory and production. Booth technology that automatically recognizes numbers in the source speech and presents them on a screen could reduce the cognitive load and improve translation quality. 

In this chapter, we present an experimental study on the properties of numbers that make them more or less challenging for the interpreter, and provide some evidence on how a technological support system influences performance.}

\shorttitlerunninghead{Simultaneous interpretation of numbers}

\maketitle
\begin{document}

%%please move the includegraphics inside the {figure} environment
%%\includegraphics[width=\textwidth]{Ch1-img1}

 
%%please move the includegraphics inside the {figure} environment
%%\includegraphics[width=\textwidth]{Ch1-img2}

 
%%please move the includegraphics inside the {figure} environment
%%\includegraphics[width=\textwidth]{Ch1-img3}
\section{Introduction}
Translation and interpreting are often called sister disciplines, but the integration of the respective activities with technology could not be more different. Computer-assisted translation is now standard practice and machine translation has become so successful that it now seems plausible to some that translators will devote most of their time to post-editing in the near future. In interpreting, by contrast, technological support is scarce, except for electronic devices used for terminology support in the booth \citep{Fantinuoli2012} or for taking notes in consecutive interpreting and hybrid modes \citep{Orlando2014,Orlando2016,Goldsmith2017}. There are a number of reasons for this discrepancy. First, in the area of interpreting, technology and, in particular, natural language technologies, are far less developed than in the area of translation. In fact, automated interpreting systems that are claimed to be effective (such as the Google Pixel Buds) first transform spoken language into some form of written or digital code before translating and converting it back to a spoken form. Automatic interpreting therefore depends on advances made on the translation front and on the availability of accurate speech-to-text and text-to-speech software. Second, spoken language does not come in nicely packed grammatical sentences but is rife with hesitations, unfinished sequences, repairs, etc., which are much harder to handle for an automatic translation system than for a human brain. Replacing the human interpreter with a reliable automatic one will require additional progress in the analysis of human language in context. Finally, even technology-supported human interpreting develops slowly, as there is little agreement among scholars whether additional sources of information in the booth are really helpful or rather distracting for the interpreter. There is also evidence of a certain aversion to technology among interpreters \citep{CorpasPastor2016}, which is likely to delay the adoption and use of technological support for some time. 

The slow progress of technology in interpreting is due to its own set of challenges. Simultaneous interpreting is a cognitively demanding task consisting of a variety of processing tasks which have to be carried out in parallel \citep{Gile1995, Seeber2011a}. Some sub-tasks are felt by most interpreters to be particularly challenging: the interpretation of numbers, of names, of enumerations, etc. \citep{Gile2009}. This paper will focus on numbers and on the effects of (simulated) technological support for the interpretation of numbers. The main research question is whether displaying numbers on a screen in the conference room, immediately after they have been articulated by the speaker, increases the accuracy of numbers in the target text. This experimental pilot study thus aims to determine if limited technological support, which would consist of automatic number recognition in the source text and the display of a numerical transcription, is at all helpful in interpreting.

\section{Numbers in interpreting}

There is a very broad consensus among interpreters that numbers are particularly difficult to interpret. Yet, research on the topic of interpreting numbers is rather limited \citep[287]{Mead2015}. Starting with \citet{Braun1996}, several experimental studies have been conducted on the success rate of number rendition in interpreting, showing that overall performance is relatively poor, both in professionals and in student interpreters. In \citet{Braun1996}, for instance, 12 students obtained a mean error score of close to 70\% while simultaneously interpreting numbers contained in speeches. \citet{Mazza2001} reports slightly better performances by 15 students, but the mean error rate in her study is still in the 45--50\% range. The findings of \citet{Pinochi2009}, based on interpretations by 16 students, are fairly consistent with Mazza’s result. Pinochi compares interpretations from two different source languages (English and German) into one target language and finds that error rates are nearly identical (ca. 40\%), even though interpreters have to overcome additional challenges due to differences in the syntax of numbers between German and Italian.

\citet{Korpal2016} compares student performances with performances by professional interpreters for slow and fast delivery rates. He finds that, although professionals obtain lower error rates than students, nearly 30\% of numbers in the interpretation are either wrong or absent altogether in the slower delivery rate. The error rate jumps to 43\% for the highest delivery rate. Among students, error rates are in line with Mazza’s findings: 44\% for the slow delivery rate and 56\% for the fast one. Timarová’s experimental study of 28 professional interpreters yields an error rate for numbers of approximately 40\% \citep{Timarova2012}. In a corpus study reported in \citet{Collard2017}, error rates for numbers in interpretations collected in the European Parliament are close to 18\%. One possible, but unverifiable, explanation for the discrepancy between the experimental and the corpus results is the presence of a booth colleague who takes down the numbers for the interpreter ensuring the turn. The numbers can then simply be read off the notebook. In an experimental setting, interpreters perform in isolation, and even though they are usually allowed to take down numbers themselves, this is obviously more difficult than when assistance is provided by a colleague in the booth. In any case, even in naturalistic data, close to one out of five numbers of the source text is rendered incorrectly or omitted in the target language. Based on this data, technological support could be helpful in the booth, and it has the potential to reduce the number of errors and omissions.

Scholars have identified several reasons why interpreting numbers is a challenge. Numbers lack a conceptual representation \citep{Timarova2012,Seeber2015} and are, therefore, not embedded in a conceptual structure allowing interpreters to anticipate them. This lack of predictability of numbers is widely recognised as an obstacle to their interpretation \citep{Braun1996,Mazza2001,Pinochi2009,Mead2015}. Numbers are highly informative \citep{Alessandrini1990}, as every component of a number is a meaningful unit representing only one particular meaning. This prevents interpreters from using strategies such as paraphrasing or reformulation \citep{Pinochi2009}. Numbers also usually lack redundant material, which makes them more informative \citep{Gile1995,Seeber2015}. Source texts with high information density are known to increase cognitive load in interpreters. As hypothesized by \citet{Pinochi2009}, differences in number syntax between source and target language can exacerbate this load, as interpreters not only have to render each numeral unit correctly but also swap the order of some of the units (e.g. between English and German).

One of the universally recommended strategies is note-taking \citep{Setton1999,Jones2002,Mead2015}: interpreters are advised to stop the delivery of the target text as soon as they hear a number, write it down on a notepad in the booth and read it off while starting up the delivery again. The findings of \citet{Mazza2001} seem to support the hypothesis that interpreting is more accurate in cases where interpreters jotted down the number. Without a notepad, shortening the ear-voice span (\textsc{evs}) and changing the listening strategy seem to be the most effective coping strategies: \citet{Setton1999}, for instance, observes that errors typically occur when the \textsc{evs} is more than 3--4 seconds. Following \citet{Seleskovitch1975}, \citet{Pinochi2009} advocates a switch from intelligent hearing, i.e. taking into account the context to draw inferences, to literal hearing, i.e. paying attention to the item in isolation.

Assistance by the booth colleague in writing down numbers and visual input provided by the speakers, such as a copy of the speech to be used in the booth, or the projected presentation slides, are said to be beneficial \citep{Mead2015}. \citet{Lamberger-Felber2001} reports a significant increase of number and name accuracy (53\% to 68\% fewer errors, depending on the source speech) in an experiment when interpreters are provided the text of the speech in the booth, compared to when they do not have the text at their disposal. Accuracy is highest when, in addition, they are given time to prepare the text they are supplied with. It is to be noted, however, that this is the combined accuracy for numbers and names together. Even for the condition without text, high accuracy rates are obtained (mean of 85.7\%). This seems to suggest that names cause significantly fewer problems in simultaneous interpreting than numbers.\largerpage[-1]

While assistance and visual input are likely to boost performance, they are beyond the control of the interpreter. Technological support could solve that problem: if interpreters could rely on technology that systematically displays numbers as they are pronounced, it could improve the accuracy of the numbers they deliver. Currently, limited applications exist in conference rooms with voting systems, where the results of votes are displayed on a screen in the booth, but the targeted use of natural language processing applications could make it possible in the near future to extract numerical information from online speeches. 

\section{Technology in interpreting}\largerpage[-1]

Technology has always been essential to simultaneous interpreting, with audio equipment and booth consoles providing the communication backbone for it to occur. More recently, remote interpreting has been making forays into the profession. \citet{Fantinuoli2018} categorizes these technologies as primarily setting-ori\-ented, since they determine the external conditions in which interpreting takes place. Process-oriented technologies, on the other hand, are designed to support the interpreter in the various phases and processes of interpreting itself, e.g. for the acquisition, organization and retrieval of information, both before and during an assignment. Such technologies aim to directly influence the interpreting process, its associated cognitive load and the quality of its outcome. As such, they are the defining components of computer-assisted interpreting (\textsc{cai}).

Currently, existing \textsc{cai} tools are mostly focused on terminological support, whether in preparing for an assignment or for access in the booth. This focus is not surprising as domain-specific terms are an important obstacle to interpreting quality, and \textsc{cai} tools have the potential of helping interpreters use them more accurately and consistently. Recent studies \citep{Will2015,Fantinuoli2017a,Costa2018} have surveyed existing \textsc{cai} tools for terminology management, and determined relevant criteria to evaluate them. For a knowledge management tool to be practical in the booth, an important requirement is that it allows the interpreter to access reference material quickly and with as little additional cognitive load as possible. This can be achieved with good knowledge representation, clear presentation and ergonomic operation, and good search algorithms.

It is essential that relevant information can be retrieved fast, i.e. within the ear-voice span. Automatic Speech Recognition (\textsc{asr}) has the potential of speeding up the look-up process and solving the cognitive effort and latency of manual querying. Technological advances in the field have been rapid in recent years, especially since the introduction of neural networks \citep{Yu2016}. Given the current state of the art in \textsc{asr} and its foreseeable progress, it seems to be a matter of time before this technology is used in \textsc{cai} tools to support interpreters with terminology look-up, and/or with information-dense content like numbers, as explored in \citet{Fantinuoli2017b}. This paper will focus on the latter of these aspects. 

While the development and adoption of \textsc{cai} tools has been limited, scientific research on the impact of their use has been even scarcer. The main contribution of this work is that it experimentally evaluates the potential impact of \textsc{asr}-driven \textsc{cai} support that displays numbers on-screen in real time.

\section{Experiment}

The aim of this experiment is twofold: to determine if limited technological support can improve the accuracy of interpreted numbers, and how this improvement breaks down over different number and error types. In the following sections, we describe the system used as a proxy for automatic number support, the participants, selected speeches and the distribution of numbers in them, the experimental setup and the evaluation parameters.

\subsection{A proxy system for automatic number support}
An ideal system for number support during simultaneous interpretation would consist of three components:

\begin{enumerate}
\item \textsc{asr} that can transform an incoming speech signal into text quickly, accurately and without the need for being tuned to a specific speaker or accent 
\item software to isolate numbers from the text in meaningful units 
\item a way of ergonomically presenting those numbers to the interpreter
\end{enumerate}

Since no such system existed at the time this study was conducted, a mock-up system was used to simulate the desired behavior. Microsoft PowerPoint presentations were prepared ahead of time based on the speech transcripts, containing one slide per number in the speech. \figref{fig:01:1} shows an example slide as used in the experiment. Numbers were presented in a numerical format with spacing between multiples of a thousand, and formatted in a large fixed-width font. The two previous numbers, if available, were displayed above the focus number, so that numbers in rapid succession would stay accessible longer. During the experiment, the presentation was shown on a big screen in the conference room, and slides were manually advanced immediately after a number had been fully pronounced, i.e. simulating an automatic system with minimal latency.

\begin{figure}
\fbox{\parbox[][3cm][t]{4cm}{%
        \raggedleft
        \vfill
        \sffamily 7,3\\
               256 150\\
                  1990\vfill
     }\hspace{1cm}
}
%%please move the includegraphics inside the {figure} environment
%%\includegraphics[width=\textwidth]{Ch1-img4}
\caption{Example of the mock-up technological support, showing the three most recent numbers that have been pronounced by the speaker (Microsoft PowerPoint slide). New numbers are added to the bottom of the slide, with old numbers shifting upwards. A maximum of three numbers is shown.\label{fig:01:1}}
\end{figure}

\subsection{Participants}

The experiment was performed with ten interpretation students enrolled in the postgraduate program for Conference Interpreting at Ghent University with\linebreak Dutch as an A-language. Seven of them had completed a 4-year applied linguistics program with a focus on French, the three others with a focus on German, making them proficient at the C2 level according to the \textsc{CEFR} framework \citep{Council2001}. At the time of the experiment, all participants had received 5 weeks of simultaneous training at the postgraduate level and were taking additional \textit{retour} classes for the source languages of the experiment. They were all graduates of a master program in interpreting that offers an introduction to simultaneous interpreting and limited practice. Participants were aged between 22 and 27 years and 9 of them were female. 

\subsection{Speeches}

Four experimental speeches of similar difficulty and length were prepared, with parallel versions in French and German. The average text length for French was 1121 words, the German texts conveyed the same content with almost 10\% fewer words, at an average of 1022 per text. German compounds are written as a single word, which largely explains this discrepancy. The texts dealt with diverse topics (Amazon, child labour, inheritance law and natural disasters), and specific terminology was provided to the participants before the start of each speech.

Like in the experiments of \citet{Braun1996} and \citet{Mazza2001}, the start of each speech (150 words) contained no numbers, to allow participants to get accustomed to the experimental conditions. After that, 20 numbers occurred at random intervals. They were equally distributed over four categories, with each text containing exactly 5 instances of each category: simple whole numbers (e.g. 87 or 60\,000), complex whole numbers (e.g. 387 or 65\,400), decimals (e.g. 28.3) and years (e.g. 2012). We distinguished between simple and complex whole numbers based on the number of meaningful units rather than the number of digits, since large numbers can be conceptually simple (e.g. 1 million contains only 2 meaningful units). In this study, numbers containing 3 or more meaningful units were considered complex.

\subsection{Setup}
Participants were required to interpret into their A-language: Dutch. The experiment was conducted in two sessions: one for French with 7 participants, and one for German with 3 participants. There was no overlap between the two participant groups. Offering two source languages created an opportunity to check the results for the influence of number syntax: German and Dutch are both “unit-decade” languages (61 is \textit{eenenzestig} in Dutch: ‘one-and-sixty’), whereas French is a “decade-unit” language (\textit{soixante-et-un} ‘sixty and one’). In interpreting from French into Dutch, the order of certain units inside the number needs to be changed, whereas if German is the source language, no such changes are required.

Before the experiment, participants were informed that numbers would occur with high frequency in the speeches, and they were familiarized with the mock-up technological support. Each speech was then read by a near-native, and recorded digitally, along with all interpretations into Dutch. The first two texts were interpreted without simulated technological support, whereas it was available for the last two. The rate of delivery, a potential problem trigger for interpretation quality \citep{Gile1995}, was within the optimal range for interpreters: an average of 121 words per minute for French, and 101 words per minute for German. This difference in rate can be partly explained by the higher information density of German words, but the German speaker also took more time to deliver speeches with identical content to their French counterparts: an average of 608 seconds, 9.4\% longer than the 556 seconds for French. Within the same language, the delivery rate for speeches interpreted without technological support was slightly slower than those interpreted with support: an average of 115 and 128 words per minute for French, and 100 and 102 words per minute for German, respectively. Any benefits from the technological support can therefore not be attributed to a slower rate of delivery.\largerpage

After the experiment, a questionnaire polled whether the participants had found the technological support helpful or distracting, how long it took to get used to it, and what they would change about it.

\subsection{Evaluation}
Given the focus of this study on interpreting numbers, the recordings were analyzed to produce systematic records for each number. These include a transcript of the number stimulus in the source speech and the provided interpretation, which allows the performance to be coded: has the number been interpreted correctly, and if not, what kind of error was made? For the categorization of errors, we follow \citet{Pinochi2009}, which in turn was adapted from \citet{Braun1996}:

\begin{description}
\item[Omission:] the number is missing or has been replaced by a general expression (e.g. \textit{47} becomes \textit{many}).
\item[Approximation:] the order of magnitude is correct, but the number has been\linebreak rounded up or down (e.g. \textit{47} becomes \textit{around 50}). These adaptations can be viewed as interpretation tactics rather than mistakes.
\item[Lexical mistakes:] the order of magnitude of the number is correct, but some of its components have been changed (e.g. \textit{47} becomes \textit{49}).
\item[Transposition:] all components are correct, but their order has been changed (e.g. \textit{47} becomes \textit{74}). This error can be especially frequent when the source and target language use a different number syntax (pronouncing units or tens first).
\item[Syntactic mistakes:] the order of magnitude is incorrect, even though the right components may be present (e.g. \textit{47} becomes \textit{470}).
\item[Phonological mistakes:] the error can be explained by phonological confusion in the source stimulus (e.g. \textit{14} becomes \textit{40}, a near-homophone in English).
\item[Other mistakes:] miscellaneous errors that do not fit any of the other categories, or numbers that combine multiple error types (e.g. \textit{47} becomes \textit{740}).
\end{description}

Additionally, the records contain information on the source language, source speech, participant, the type of number, and whether the experimental technological support was available (i.e., the independent variable). These variables allow the impact of \textsc{cai} support to be analyzed from multiple perspectives, as presented in the following section.

\section{Results and discussion}\largerpage
The experiment yielded 40 recordings of 4 speeches interpreted by 10 participants. French was the source language in 28 of them, German in 12, and half of them were interpreted with technological support for numbers. With 20 numbers per recording, a total of 800 observations was available for this pilot study.

\subsection{The impact of technological support for interpreting numbers}
Numbers are difficult to interpret, as the experimental results show. The average accuracy on numbers in the control setting, without technological support, was 56.5\% over all test subjects and languages, i.e. an average error rate of 43.5\%. The observed performances were thus completely in line with the 40--50\% error rate found in most other studies. Accuracy between individuals varied from 27.5\% to 90\%, and the interpretations from German were more accurate on average (74\% vs 49\%). This could be due to the higher similarity between the German and Dutch number systems, but the number of participants is too small to rule out individual differences as the main cause of this divergence. Another explanation is the slower rate of delivery for German than for French.

When technological support was made available, the performance on numbers improved dramatically. Average accuracy rose by 30 percentage points to 86.5\%, and the absolute number of errors dropped from 174 to 54 out of 400, an error reduction of 69\%. This finding was fairly consistent with the 53--68\% decrease reported by \citet{Lamberger-Felber2001} for numbers and names, when copies of speeches were made available to the interpreter. A paired t-test showed that the performance difference was highly significant for both source languages (p < 0.001). It should not come as a surprise that individuals that scored poorly without support benefit most from having it, at least in absolute terms of avoided errors (up to 21 fewer errors on 40 numbers). Interestingly, relative error reduction was moderate (43\%) to high (90\%) regardless of a subject’s performance in the control setting. In other words, even if an interpreter was highly competent at conveying numbers without support, he or she was able to reduce their error rate significantly when support is available.

It is obvious from these results that \textsc{cai} support for numbers has the potential of drastically reducing errors on numbers. Nevertheless, it needs to be emphasized that these results indicate the ceiling performance of what can be achieved with technological support, or with a dedicated booth mate, for that matter. The tested mock-up system has a minimal delay and its output is entirely accurate. With such a reliable system, the interpreter can choose to reduce the listening and memory efforts to focus more on production. Further research is required to test how an imperfect output would affect the distribution of effort.

Feedback from the questionnaires indicated that all participants perceived the support as helpful. Three participants reported that it took some time to get used to the system, while for the others the transition was almost immediate. Two respondents were sometimes distracted by the system because it focused their attention on numbers at the expense of other content. Future evaluations should elucidate if distraction can be reduced with increased familiarity with number support, or different ways of providing it. Specifically, adding units (e.g. \textit{km} or \textit{percent}) or concepts to the numbers could be beneficial to recall their usage in context, although this may require cues that are specific to the source language.

\subsection{The influence of number type}\largerpage
The numbers expressed in the experimental speeches were balanced over four categories: simple and complex whole numbers, decimals and dates. \figref{fig:01:2} shows the performance in terms of accuracy, categorized by these number types.

\begin{figure}
	%%please move the includegraphics inside the {figure} environment
	%%\includegraphics[width=\textwidth]{Ch1-img5}
\begin{tikzpicture}
            \begin{axis}[
                    ybar,
                    ylabel={\%},
                    xtick=data,
                    axis lines*=left,
                    ymin=0,
                    scaled y ticks=false,
                    colormap/Greys-5,
                    cycle list/Greys-5,
                    legend pos=north west,
                    xticklabels={date,decimal,complex,simple},
                    ticklabel style={font=\footnotesize\scshape},
                    legend style={font=\footnotesize,at={(0.5,-0.25)},anchor=north},
                    legend columns=-1,
                    width=\textwidth,
                    height=5cm,
                    enlarge x limits={0.1},
                    nodes near coords,
                    nodes near coords style={text=black,font=\footnotesize},
                    ]
                \addplot+[
                     fill=Greys-E,draw=none
                    ] coordinates {(0,89) (1,90) (2,82) (3,85)};
                \addlegendentryexpanded{with support}              
                \addplot+[                                     
                     fill=Greys-I,draw=none
                    ] coordinates {(0,73) (1,51) (2,32) (3,70)};
                \addlegendentryexpanded{without support}                                                                    
            \end{axis}                                                                           
\end{tikzpicture}
\caption{Average accuracy per number type in both experimental conditions\label{fig:01:2}}
\end{figure}

It can be observed that in the standard setting without support, interpreters experience most difficulties with complex and decimal numbers. Accuracy rates for these types are low at 32\% and 51\%, respectively, compared to around 70\% for both simple numbers and dates. These results corroborate previous findings. This study used the same number typology as \citet{Mazza2001}, with the exception that we distinguish whole numbers based on complexity rather than size; \citet{Pinochi2009} separates whole numbers into three categories, based on the complexity of pronouncing them. In all three studies, large or complex whole numbers are found to cause the most errors, followed by decimals, and small or simple whole numbers and dates are the easiest to interpret.

When interpreters have access to technological support, the differences in error rate per number type almost disappear. Overall, we see that each type benefits from it. As expected, the largest gain is for complex whole and decimal numbers, with absolute error reductions of 50 and 39\%, respectively (significant at $p < 0.01$). Even for the simpler number categories there are significant improvements ($p < 0.05$). With technological support, accuracy is almost identical across number types, which suggests that the remaining errors are due to factors other than the complexity of the number. A detailed analysis of the remaining errors is still to be made.   

\subsection{Error type analysis}\largerpage[1]

A total of 228 errors was observed in the experiment. \figref{fig:01:3} presents them separated by type.

\begin{figure}
	%%please move the includegraphics inside the {figure} environment
	%%\includegraphics[width=\textwidth]{Ch1-img6}
\begin{tikzpicture}
            \begin{axis}[
                    ybar,
                    ylabel={\%},
                    xtick=data,
                    axis lines*=left,
                    ymin=0,
                    trim axis left,
                    trim axis right,
                    scaled y ticks=false,
                    colormap/Greys-5,
                    cycle list/Greys-5,                    xticklabels={other,phonological,syntactic,transposition,lexical,approximation,omission},
                    ticklabel style={font=\footnotesize\scshape},
                    x tick label style={rotate=45,anchor=east,xshift=-.5cm,yshift=\baselineskip,tickwidth=0pt},
                    legend style={font=\footnotesize,at={(0.5,-0.55)},anchor=north},
                    legend columns=-1,
                    width=\textwidth,
                    height=5cm,
                    enlarge x limits={upper,abs=2cm},
                    ymax=30,
                    restrict y to domain*=0:35,
                    visualization depends on=rawy\as\rawy, % Save the unclipped values
                        after end axis/.code={ % Draw line indicating break
                            \draw [ultra thick, white, decoration={snake, amplitude=1pt}, decorate] (rel axis cs:0,1.05) -- (rel axis cs:1,1.05);
                         },
                    nodes near coords={\pgfmathprintnumber{\rawy}% Print unclipped values
                    },
                    nodes near coords style={text=black,font=\footnotesize},
                    clip=false
                    ]
                \addplot+[
                     fill=Greys-E,draw=none,xshift=-1cm
                    ] coordinates {(0,1) (1,1) (2,4) (3,0) (4,5) (5,4) (6,39)};
                \addlegendentryexpanded{with support}              
                \addplot+[                                     
                     fill=Greys-I,draw=none,xshift=-1cm
                    ] coordinates {(0,5) (1,2) (2,3) (3,4) (4,15) (5,39) (6,106)};
                \addlegendentryexpanded{without support}                                                                    
            \end{axis}                                                                           
\end{tikzpicture}	
\caption{Error count per type in both experimental conditions\label{fig:01:3}}
\end{figure}

By far the most frequently occurring error is omission, regardless of the availability of technological support. Technological support significantly reduces the number of omissions ($p < 0.05$), but omissions remain frequent and their relative weight in the total error load increases from 61 to 72\%. We see two possible explanations for this. Some numbers come in information-dense sections of a speech, and omitting them may be a necessity to limit cognitive load or to avoid increasing \textsc{evs}. In such cases, even having the number available on screen would not help in reproducing it in context. Second, it could be that technological support causes complacency or confusion when numbers are heard. If an interpreter reduces the listening effort and relies on a support system to receive the number, adequate context may be lacking to convey its meaning.

Approximations are often used in the control setting, but only when the source stimulus was a decimal or a complex whole number. Approximating is a useful strategy when numbers have not been entirely understood, or when they are too large to fit in working memory. With the addition of technological support, this strategy is used almost 10 times less often, since the tasks of comprehension and working memory are effectively solved. The error reduction is significant at $p < 0.001$.

Lexical mistakes, the third most frequent error in the control setting, occur three times less often when support is available. The other four error categories do not occur frequently in either of the settings. Differences between the two experimental settings are not significant or lack support.

The error distributions are in line with the findings of \citet{Mazza2001} and \citet{Pinochi2009}, who also found omissions and approximations to be most frequent, in that order.

\section{Conclusions and future work}\largerpage[1]

This paper presented an experimental pilot study of the potential impact of booth technology that supports the interpretation of numbers. Our mock-up system simulates technology that automatically recognizes numbers in the source\linebreak speech and presents them on a screen in the conference room, in order to reduce the cognitive load and improve translation quality. 

Technological support improves overall accuracy on numbers from 56.5 to 86.5 percent, reducing the amount of errors by two thirds. The improvement is statistically significant for all participants. Technological help is most helpful in reducing errors on complex numbers and decimals, the two categories that are most often interpreted incorrectly. Omissions are the most frequent error, followed by approximations. The occurrence of the latter drops by almost 90 percent when support is available.

Since the experiment was performed with students, the results are not readily applicable to professional interpreters. Even though the outcome of the experiment clearly shows the potential of \textsc{cai} support for numbers, it must be emphasized that our experimental design is not based on automatic recognition of numbers in speech. Automatic Speech Recognition might not achieve perfect recognition and minimal latency. Therefore, our results describe the ceiling performance that could be achieved with such an ideal system. Further studies should be carried out on how interpreters deal with discrepancies between auditory input from a speaker and visual input from an automatic recognition system, increased delay or different modes of presentation. Further research should also focus on the rendition of items used in combination with numbers, as it is known that interpreters tend to direct so many of their attentional resources to numbers that errors also frequently occur in the context of numbers \citep{Gile2009}.

\sloppy
\printbibliography[heading=subbibliography,notkeyword=this]

\end{document}
